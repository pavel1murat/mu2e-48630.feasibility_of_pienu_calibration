%%%%%%%%%%%%%%%%%%%%%%%%%%%%%%%%%%%%%%%%%%%%%%%%%%%%%%%%%%%%%%%%%%%%%%%%%%%%%%
\section{Comparison to the analysis by Purdue group}

\begin{itemize}
\item
  The analysis by Purdue group claims that the analysis used procedure introduced
  in \cite{KRZYSZTOF}. However the analysis also uses the DIF event weights
  varying from one event to another and defined by the individual muon decay probability.
  That is a mistake.
\item
  For essentially the same cuts, the background estimate is about x10 lower
\item
  For the 3mm degrader, the quoted signal yield (assuming it corresponds to a 2 MeV window)
  is about 1.9 times higher than in the present analysis, 9.2e-13/POT vs 4.8e-13/POT.
\item
  For the 4 mm degrader, the quoted signal yield (assuming the same window) 
  is about 1.4 times higher , 4.4e-13/POT vs 3.2e-13/POT
\item
  at the time the group used the technique of \cite{KRZYSZTOF}, the standard implementation
  had a bug - the technique has been implemented for $\mu^-$, but not for $mu^+$'s.
  The authors didn't respond to the question of whether of not
  they made additional changes to the code introduced in \cite{KRZYSZTOF}.
  The analysis doesn't present validation of the improvement in the statistical power due
  to using \cite{KRZYSZTOF}. 
\end{itemize}
%%% Local Variables:
%%% mode: latex
%%% TeX-master: "mu2e-xxxxx"
%%% End:
