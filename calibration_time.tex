%%%%%%%%%%%%%%%%%%%%%%%%%%%%%%%%%%%%%%%%%%%%%%%%%%%%%%%%%%%%%%%%%%%%%%%%%%%%%%%
\section{Time needed for the momentum scale calibration }

Calibration with \piplusenu\ allows to determine the momentum scale of the experiment.
However, this determination is done in a magnetic field B, for which positrons
from \piplusenu\ get within the detector acceptance, i.e. B $\sim$ 0.7 T.

\begin{itemize}
\item
  assume the peak is reconstructed at momentum $p_0$. Momentum scale correction then is
  $$
                SF ~=~ \frac{69.8 MeV - \delta p} {p_0}
  $$
\end{itemize}
where $\delta p$ is the average ionization loss of a 70 MeV/c positron.

In this section we make a simplifying assumption that the background under the \piplusenu\
peak doesn't affect the accuracy of the peak position determination and under that assumption
estimate the time needed for the calibration.

\begin{itemize}
\item 
  required accuracy of the momentum scale calibration $@$100 MeV/c: $\sigma_P/P$ < 100 keV/c
\item
  for an estimate, require an accuracy of the peak position determination
  twice as good, $\sim$ 50 MeV/c.
  FWHM of the \piplusenu\ momentum peak of 1 MeV/c translates into an estimate of
  $\sigma_P \sim 500$ keV/c, and 
  $$,
  \frac {500 {\rm ~keV/c}}{\sqrt N} <  50 {\rm ~keV/c} ~~
  $$
  or $N > 100$ events in the peak. We conservatively assume that the momentum scale calibration
  requires 1000 reconstructed \piplusenu\ events.
\item
  For a yield of $10^{-13}$ events/POT, collecting 1000 events requires $10^{16}$ protons on target.
  In one-batch mode, an average expected pulse intensity is $1.6 \times 10^7$, and
  an average  pulse rate of 1$.6 \times 10^5$ pulses/sec give the rate of $2.5 \times 10^{12}$ protons/sec.
\item
  Assuming running at 10\% of nominal beam intensity and the data collection efficiency of 50\%,
  collecting 1000 reconstructable \piplusenu\ events would require
  $10^{16}/(1.25 \times 10^{11}) \sim 10^5$ seconds, or about one day of running.
\item
  running at 10\% of nominal beam intensity with the digitization starting at 200 ns results in 
  the total number of background hits per microbunch of about 200, so the pileup at T>300 ns
  should not be a problem.
\end{itemize}

Note, that the considerations above do not touch the "extrapolation" of the momentum scale up
to the momentum of 100 MeV/c.

%%% Local Variables:
%%% mode: latex
%%% TeX-master: "mu2e-xxxxx"
%%% End:
