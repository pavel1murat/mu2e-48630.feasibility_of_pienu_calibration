%%%%%%%%%%%%%%%%%%%%%%%%%%%%%%%%%%%%%%%%%%%%%%%%%%%%%%%%%%%%%%%%%%%%%%%%%%%%%%%
\section{Time needed for the momentum scale calibration }

Calibration with \piplusenu\ allows to determine the momentum scale of the experiment.
However, this determination is done in a magnetic field B, for which positrons
from \piplusenu\ get within the detector acceptance, i.e. B $\sim$ 0.7 T.

In this section we make a simplifying assumption that the background under the \piplusenu\
peak doesn't affect the accuracy of the peak position determination and under that assumption
estimate the time needed for the calibration.

\begin{itemize}
\item 
  required accuracy of the momentum scale calibration $@$100 MeV/c: $\sigma_P/P$ < 100 keV/c
\item
  assuming, for simplicity, that the systematic uncertainty at 50 MeV/c is negligible,
  extrapolation of the momentum scale from 70 MeV/c to 100 MeV/c will result in the
  systematic uncertainty
  $$
  \sigma_P \bigg\rvert_{1.0 T} = 3/2 \times \sigma_P  \bigg\rvert_{0.7 T}
  $$
  so an accuracy of the \piplusenu\ peak position determination should be < 100 MeV /1.5, say, $\sim$ 50 MeV/c.
  Starting from FWHM of the \piplusenu\ momentum peak of 1 MeV/c, we assume $\sigma_P \sim 500$ keV/c, and 
  $$,
  \frac {500 {\rm ~keV/c}}{\sqrt N} <  50 {\rm ~keV/c} ~~
  $$
  where N is the number of events in the peak. This gives $N > 100$ events in the peak.
  We conservatively assume that the momentum scale calibration requires 1000 reconstructed \piplusenu\ events.
\item
  For a yield of $10^{-13}$ events/POT, collecting 1000 events requires $10^{16}$ protons on target.
  In one-batch mode, an average expected pulse intensity is $1.6 \times 10^7$, and
  an average  pulse rate of 1$.6 \times 10^5$ pulses/sec correspond to the rate of $2.5 \times 10^{12}$ protons/sec.
\item
  Assuming running at 10\% of nominal beam intensity and the data collection efficiency of 50\%,
  collecting 1000 reconstructable \piplusenu\ events would require
  $10^{16}/(1.25 \times 10^{11}) \sim 10^5$ seconds, or about one day of running.
\item
  running at 10\% of the nominal beam intensity in one-batch mode and with the digitization starting
  at 200 ns corresponds to the total number of background hits per microbunch of about 200,
  so the pileup at T>300 ns should not be a problem.
\end{itemize}

%%% Local Variables:
%%% mode: latex
%%% TeX-master: "mu2e-xxxxx"
%%% End:
