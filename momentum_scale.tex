%%%%%%%%%%%%%%%%%%%%%%%%%%%%%%%%%%%%%%%%%%%%%%%%%%%%%%%%%%%%%%%%%%%%%%%%%%%%%%
\section{Momentum scale calibration at B=1 T with  $\mu^+ \to e^+ \nu \nu$  and \piplusenu }

The momentum scale calibration discussed in this section relies on measurements
of the edge of the positron momentum spectrum from $\mu^+ \to e^+ \nu \nu$ at B=0.5 T
and the measurement of the \piplusenu\ peak at B = 0.7 T.

Each of the two measurements allows to determine the momentum scale at the
respective value of the magnetic field. Combining them allows to extrapolate
the momentum scale to the nominal field, B = 1.0 T.
The idea is illustrated by Figure~\ref{figure:momentum_scale}

Momentum scale calibration scheme assumes that the magnetic field is set based
on reading an NMR probe, and the scale correction can be effectively encapsulated
by re-scaling the value of the nominal magnetic field. 

\begin{figure}[H]
  \begin{tikzpicture}
    \node[anchor=south west,inner sep=0] at (0,0.) {
      % \node[shift={(0 cm,0.cm)},inner sep=0,rotate={90}] at (0,0) {}
      \makebox[\textwidth][c] {
        \includegraphics[width=0.8\textwidth]{pdf/momentum\_scale\_calibration}
      }
    };
    % \node [text width=8cm, scale=1.0] at (14.5,0.5) {$\mu_B$, expected background mean};
    % \node [text width=8cm, scale=1.0, rotate={90}] at (1.5,7.5) { $S_{D}$, ``discovery'' signal strength  };
  \end{tikzpicture}
  \caption{
    \label{figure:momentum_scale}
    momentum scheme calibration
  }
\end{figure}

The measured position of the Michel spectrum edge can be written as follows 
$$
    P_{Michel} = 52.8 - \Delta{P_{EL}}{50} + \delta_{scale}
$$
, where  $\Delta P_{EL}$ is the term corresponding to the energy losses not accounted
by the track reconstruction, $P_{Michel}$ is the measured position of the Michel edge,
and $\delta_{scale}$ is the momentum correction due to the momentum scale.
So at B=0.5 T the relative correction due to the momentum scale is 
$$
 \delta_{scale}/P_0 (B = 0.5 T) ~=~ \frac {P_{Michel}- 52.8 + \Delta{P_{EL}}^{50}}{P_0}
$$

, where $P_0 ~=~ 52.8 - \Delta{P_{EL}}$ is the expected  position of the Michel edge spectrum.

Similarly, a measurement of the \piplusenu\ peak position determines
$$
 \delta_{scale}/P_0 (B = 0.7 T) ~=~ \frac {P_{\piplusenu}- 69.8.8 + \Delta{P_{EL}^{70}}}{P_0}
$$
, and $\Delta{P_{EL}}^{70}$ now is the energy losses by a 70 MeV/c positron.


With the two measurements of the momentum scale performed at B=0.5 T and B=0.7 T,
the momentum scale correction at B = 1.0 T can be obtained by a linear extrapolation
$$
{\frac{\delta_{scale}}{P_0}}\bigg\rvert_{1.0 T} =  {\frac{\delta_{scale}}{P_0}}\bigg\rvert_{0.7 T} + {\frac{1.0 - 0.7}{0.7-0.5}}
  \times
( {\frac{\delta_{scale}}{P_0}}\bigg\rvert_{0.7 T} -  {\frac{\delta_{scale}}{P_0}}\bigg\rvert_{0.5 T})
$$

%%% Local Variables:
%%% mode: latex
%%% TeX-master: t
%%% End:
