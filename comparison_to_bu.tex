%%%%%%%%%%%%%%%%%%%%%%%%%%%%%%%%%%%%%%%%%%%%%%%%%%%%%%%%%%%%%%%%%%%%%%%%%%%%%%
\section{Comparisons to previous analyses} 

%%%%%%%%%%%%%%%%%%%%%%%%%%%%%%%%%%%%%%%%%%%%%%%%%%%%%%%%%%%%%%%%%%%%%%%%%%%%%%
\subsection{Comparison to BU analysis}

\begin{itemize}
\item 
  Signal yield w/o the degrader: 2.4e-12 BU (2.9 - without the momentum cuts).
  Present analysis 1.8e-12 (67.5-70 MeV/c), 30\% lower
\item
  BU number with the 3.5 mm Be degrader : (after all cuts - mu2e-5391 ,
  what are they?): $2.3 \times 10^{-12}$ / POT 
  The ratio of densities is $\sim$ 2.5 (4.5 vs 1.8 g/cm3),
  so a 3.5 Be degrader, in the 1st approximation is equivalent to a 1.4 mm Ti degrader
\item
  w/o the degrader, the background is about x10 lower:
  at 69 MeV/c it is 2e-12 per POT per 0.5 MeV for BU analysis
  and 2.4e-11 per POT per 0.5 MeV fro this analysis
\item
  Change in the MC? - the fraction of muons arriving to the DS entrance after 200 ns
  is 3\% for the BU analysis and about 30\% for this analysis
\end{itemize}


{\red
  comparison for the 3.5 mm Be degrader - Sridhar
}


%%%%%%%%%%%%%%%%%%%%%%%%%%%%%%%%%%%%%%%%%%%%%%%%%%%%%%%%%%%%%%%%%%%%%%%%%%%%%%
\subsection{Comparison to the analysis by Purdue group}


\begin{itemize}
\item
  The analysis by Purdue group claims that the analysis used procedure introduced
  in \cite{KRZYSZTOF}. However the analysis also uses the DIF event weights
  varying from one event to another and defined by the individual muon decay probability.
  That is a mistake.
\item
  For essentially the same cuts, the background estimate is about x10 lower
\item
  For the 3mm degrader, the quoted signal yield (assuming it corresponds to a 2 MeV window)
  is about 1.9 times higher than in the present analysis, 9.2e-13/POT vs 4.8e-13/POT.
\item
  For the 4 mm degrader, the quoted signal yield (assuming the same window) 
  is about 1.4 times higher , 4.4e-13/POT vs 3.2e-13/POT
\item
  at the time the group used the technique of \cite{KRZYSZTOF}, the standard implementation
  had a bug - the technique has been implemented for $\mu^-$, but not for $mu^+$'s.
  The authors didn't respond to the question of whether of not
  they made additional changes to the code introduced in \cite{KRZYSZTOF}.
  The analysis doesn't present validation of the improvement in the statistical power due
  to using \cite{KRZYSZTOF}. 
\end{itemize}


%%% Local Variables:
%%% mode: latex
%%% TeX-master: "mu2e-xxxxx"
%%% End:
